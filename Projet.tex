\documentclass[a4paper,11pt]{article}


\usepackage[utf8]{inputenc}   % un package qui permet de gérer l'encodage des caractères spéciaux.
\usepackage[T1]{fontenc}      %   permet d'utiliser tout le clavier
\usepackage[francais]{babel}  %   spécifie que l'on écrit en français
\usepackage{color}
\usepackage{hyperref}

%\usepackage[pagebackref]{hyperref}    
 
\usepackage{graphicx}
\graphicspath{ {images/} }

\usepackage[top=2cm, bottom=2cm, left=3cm, right=3cm]{geometry}   % Contrôles des marges 
\usepackage{listings}   % Pour ajouter du code

 
 % Rajouter le code en Annexe :

\lstset{ 
language=Python,     	        % choix du langage
basicstyle=10,                  % taille de la police du code
numbers=left,                   % placer le numéro de chaque ligne à gauche               
numberstyle=\normalsize,        % taille de la police des numéros
numbersep=5pt,                  % distance entre le code et sa numérotation
backgroundcolor=\color{white},  % couleur du fond 
frame=single,
breaklines=true,
basicstyle=\ttfamily,
basicstyle=\scriptsize,
keywordstyle=\color{blue},
commentstyle=\color{green},
stringstyle=\color{red},
identifierstyle=\ttfamily,
}







% Informations :

\title{ Développement d’une Application d’Assistance aux Personnes Agées pour le Robot Humanoïde NAO }
\author{ Mohamed Adibe Chemaou, Xhersika Kenga,Sebastien Roach, Marie Legrand}
\date{ Mai 2017 }





\begin{document}

\maketitle   % fait l'affichage des informations


   ~\\
   ~\\


\renewcommand{\contentsname}{Sommaire } % Pour changer le nom automatique de la table des matières
\tableofcontents
\newpage





~\\
~\\
\section{ Introduction }


	
	   
~\\
\section{Le contexte et problème }



   
     
~\\
~\\
\section{La solution }



	\subsection{Définition des scénarios }
	
	
	~\\
	~\\	
	\subsection{Architecture global du système }
	
	
		   
	~\\
	~\\	
	\subsection{ Réalisation scénario d’interaction }
	
	
		   
	
~\\
~\\
\section{ La gestion du projet  }


	    \subsection{ Planning }
	    
	    
	    
	    ~\\
	    ~\\	    
	    \subsection{ Organisation }
	    
	    
	    ~\\
	    ~\\	    
	    \subsection{ Difficultés et solutions} \label{ret}
	    
	    
	    
	    
	    ~\\
	     
			    
\section{ Conclusion }




		   
		   
\appendix % signifie que l'on passe sur les pages annexes		   
		   

		   
\newpage		   
~\\ 		   			    





\end{document}
