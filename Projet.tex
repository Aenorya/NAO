\documentclass[a4paper,11pt]{article}


\usepackage[utf8]{inputenc}   % un package qui permet de gérer l'encodage des caractères spéciaux.
\usepackage[T1]{fontenc}      %   permet d'utiliser tout le clavier
\usepackage[francais]{babel}  %   spécifie que l'on écrit en français
\usepackage{color}
\usepackage{hyperref}

%\usepackage[pagebackref]{hyperref}    
 
\usepackage{graphicx}
\graphicspath{ {images/} }

\usepackage[top=2cm, bottom=2cm, left=3cm, right=3cm]{geometry}   % Contrôles des marges 
\usepackage{listings}   % Pour ajouter du code

 
 % Rajouter le code en Annexe :

\lstset{ 
language=Python,     	        % choix du langage
basicstyle=10,                  % taille de la police du code
numbers=left,                   % placer le numéro de chaque ligne à gauche               
numberstyle=\normalsize,        % taille de la police des numéros
numbersep=5pt,                  % distance entre le code et sa numérotation
backgroundcolor=\color{white},  % couleur du fond 
frame=single,
breaklines=true,
basicstyle=\ttfamily,
basicstyle=\scriptsize,
keywordstyle=\color{blue},
commentstyle=\color{green},
stringstyle=\color{red},
identifierstyle=\ttfamily,
}







% Informations :

\title{ Développement d’une Application d’Assistance aux Personnes Agées pour le Robot Humanoïde NAO }
\author{ Mohamed Adibe Chemaou, Xhersika Kenga,Sebastien Roach, Marie Legrand}
\date{ Mai 2017 }





\begin{document}

\maketitle   % fait l'affichage des informations


   ~\\
   ~\\


\renewcommand{\contentsname}{Sommaire } % Pour changer le nom automatique de la table des matières
\tableofcontents
\newpage





~\\
~\\
\section{ Introduction }


	
	   
~\\
\section{Le contexte et problème }



   
     
~\\
~\\
\section{La solution }



	\subsection{Définition des scénarios }
	
	
	~\\
	~\\	
	\subsection{Architecture global du système }
	
	
		   
	~\\
	~\\	
	\subsection{ Réalisation scénario d’interaction }
	
	\subsubsection{Appel météo}
	\paragraph {}Ce scénario était un des premiers scénarios réalisés. Le principe de ce fonctionnalite est très simple: Chaque fois que la personne âgée demande à NAO de savoir les conditions métrologiques actuelles, NAO lui donne toutes les informations nécessaires.
En plus il donne des conseils pour la manière dont la personne doit s'habiller, par rapport au météo.
\paragraph{} L'aspect technique de ce scénario est le suivant:
\newline Dans un box Python, on lit un fichier json qui contient toutes les donnes de météo courantes. Ce fichier est lu depuis un service météo en ligne , l'OpenWeatherAPI.
En premier lieu, avant d'avoir commencé réaliser notre code, on a fait une inscription sur le site du notre fournisseur des donnes. Après cette inscription on récupère un user id qui est indispensable pour les prochaines étapes de réalisation. 
\paragraph{} Dans un premier temps a cree une fonction $url_ builder(city id)$. Dans cette fonction on fait la génération d'une URL valide pour accéder nos donnes météo qui se trouve dans le site du OpenWeatherAPI. Notre lien complet pour accéder au fichier json est constitué de l'adresse api, ou on ajoute le mode de récupération des donnes(dans notre cas jSon) et notre user id.
\paragraph{} Apres avoir généré notre URL, il était nécessaire de récupérer notre fichier jSon .Pour faire cela on a cree une fonction $datafetch(URL)$, qui prenant en paramètre l'URL de notre fichier, retourne toutes les donnes que ce fichier contient. Les modules python nécessaire pour réaliser cette tâche étant : json et urllib2.
\newline On utilise une fonction du module json appele json.loads qui servent à decoder notre fichier json et le charger en memoire. 
\paragraph{} La prochaine étape de réalisation de notre scénario etait d'organiser toutes les donnes brut recupere de notre fichier json, pour ensuite être capable à les transmettre à notre robot, de manière qu'il puisse les communiquer sous forme orale a la personne. La fonction qui sert à faire l'organisation des données et $data_organizer$ qui prend en paramètre les donnes retournées par notre fonction précédent $data_fetch$. Il s'occupe de la création d'un dictionnaire python, qui a comme clés la ville, température maximale, minimale etc. Il renvoye ce dictionnaire à la fin.
\paragraph{}L'étape prochaine, qu'on peut qualifier comme étant la plus essentielle, consiste à appeler une fonctionnalite de NAO et Coreographe qui ete très importante et frequentment utilise dans notre projet, le module ALTextToSpeech. Il renvoit du texte au moteur d'articulation de NAO. Pour utiliser ce module on se sert d'une ligne très simple: tts = ALProxy("ALTextToSpeech", adresseDuRobot, portDuRobot). TTS(text-to-speech)est une variable instance de la classe ALProxy ,fourni par Coreographe.Cette classe permet de se connecter a un robot NAO pour nottamment lui envoyer des chaines de charactere a dire oralemment.
\newline Notre fonction $data_output$ prend alors en entree notre dictionnaire cree precedamment. On utilise la variable tts qu'on appele avec tts.say, la fonction du module ALTextToSpeech qui prend en paramètre un texte et traduit ce dernier en speech de notre robot.
\paragraph{} Dans notre fonction de classe $onInput_onStart$ on appele :(data organizer(data fetch(url builder()))). Cette ligne permet de lier toute les fonction cree dans notre petit box python.
\newline A ce stade  NAO est donc capable de dire clairement la météo actuelle quand la personne lui demande!

	
~\\
~\\
\section{ La gestion du projet  }


	    \subsection{ Planning }
	    
	    
	    
	    ~\\
	    ~\\	    
	    \subsection{ Organisation }
	    
	    
	    ~\\
	    ~\\	    
	    \subsection{ Difficultés et solutions} \label{ret}
	    
	    
	    
	    
	    ~\\
	     
			    
\section{ Conclusion }




		   
		   
\appendix % signifie que l'on passe sur les pages annexes		   
		   

		   
\newpage		   
~\\ 		   			    





\end{document}
